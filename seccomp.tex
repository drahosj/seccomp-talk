\documentclass[handout]{beamer}

\usepackage[utf8]{inputenc}

\usepackage{listings}
\lstset
{ 
    language=C,
    numbers=left,
    basicstyle=\tiny
}

\usetheme{Warsaw}
\mode<presentation>{}


\setlength{\parskip}{\baselineskip} 

\usepackage[style=verbose,backend=biber]{biblatex}
\addbibresource{seccomp.bib}

\usepackage{tikz}
\usetikzlibrary{positioning}

\usepackage{bytefield}

\title{Kernel Hacker Stuff - Seccomp}
\author{Jake Drahos}


\AtBeginSection[]{
  \begin{frame}
  \vfill
  \centering
  \begin{beamercolorbox}[sep=8pt,center,shadow=true,rounded=true]{title}
    \usebeamerfont{title}\insertsectionhead\par%
  \end{beamercolorbox}
  \vfill
  \end{frame}
}


\begin{document}
\begin{frame}
 \titlepage
\end{frame}

\section{Overview}
\subsection{Intro}
\begin{frame}{The SECure COMPuting kernel facility}
    Seccomp is a kernel facility providing a one-way transition into ``secure computing'' mode,
    restricting which parts of the kernel API are accessible to the process. \footcite{kerneldoc}
\end{frame}

\subsection{Disclaimer}
\begin{frame}{SECCOMP IS NOT A SANDBOX}
  (this disclaimer itself is paraphrased from the official documentation)
  
  Seccomp, on its own, is neither a security feature nor a sandbox!
  
  \pause
  However, it can be (and is) used to implement those things.
  
  \pause
  While it isn't something you just ``turn on and get security'', it is important (or
  at least interesting) to understand the kernel facilities that support useful security features).
  
\end{frame}

\section{Operating Systems 101}
\subsection{Process Isolation}
\begin{frame}{Processor Execution Modes}
\begin{itemize}
 \item Modern (post-1980s) processors are designed to support the concept of ``process isolation''
 via multiple privilege levels
 \item User/process privilege level sees the machine as only its abstract ISA
    \begin{itemize}
%        \item ``Simplified'' - flat memory layout, no weird IO, no special instructions...
        \item ``Privilege level'' in an ISA is a different concept that an OS user's privilege level
    \end{itemize}
 \item Any interaction with the outside world must be done by the kernel on behalf of the process!
\end{itemize}
\end{frame}

%
%\begin{frame}
% What does a process see?
% \begin{itemize}
%  \item Its mapped memory (but has no ability to modify the mapping)
%  \pause
%  \item The processe's own instruction stream
%  \pause
%  \item Processor state (registers, condition flags, CSRs...)
% \end{itemize}
%\end{frame}

%\begin{frame}
% Notable omissions:
% \begin{itemize}
%  \item Most, if not all, concepts of IO
%  \item Other process memory
%  \item Kernel objects (synchronization locks, message queues, ...)
%  \item MMU mapping
% \end{itemize}
% 
% \pause
% The kernel must provide services to interact with other processes and the real world. 
%\end{frame}

\subsection{System Calls}
\begin{frame}{The syscall instruction}
ISAs provide a system call instruction (\texttt{int 80h, syscall, ecall, t 0x6d,} etc.)

\pause
What the processor does when a syscall instruction executes:
\begin{enumerate}
 \item Save execution state (PC, SP, GP registers, condition...)
 \item Switch to kernel-mode memory map
 \item Increase privilege level
 \item Begin executing instructions at a configured address
\end{enumerate}

The address where kernel-mode execution begins is the syscall entry handler

\end{frame}

\begin{frame}{System Calls as Function Calls}
 A syscall API can be standardized just like any other library API.

 \pause
 Processes can execute a syscall by:
 \begin{enumerate}
  \item Setting up any buffers (allocate/copy memory)
  \item Place arguments (pointers, flags, sizes...) where the ABI dictates
  \item Execute the syscall instruction
 \end{enumerate}
 
 Kernel looks at arguments, info about the process (user, OS permissions...) and decides if/how
 to fulfill the request.
\end{frame}

\section{The Linux Syscall Interface}

\begin{frame}
 Syscalls ultimately provide the only* mechanism by which processes can do
 anything interesting.
 
 Syscalls are documented in man(2), plus kernel documentation for ABI details
 \pause
 
 * getting creative with memory mapping (and evil x86 in/out instructions) can allow
 direct access to IO (or shared memory), but require syscalls to set up.
\end{frame}

\subsection{Classic Syscalls}
\begin{frame}{Traditional Unix IO}
Simple, standard file operations
   \begin{itemize}
    \item open()
    \item close()
    \item read()
    \item write()
   \end{itemize}
 
 \pause
 (actually the *at versions tend to be used by glibc because reasons)
\end{frame}

\begin{frame}{Process Management}
 Unix process lifecycle managed by syscalls
 \begin{itemize}
  \item \texttt{fork()}
  \item \texttt{execve()}
  \item \texttt{wait()}
 \end{itemize}
\end{frame}

\begin{frame}{Memory map manipulation}
Some syscalls allow a process to manipulate its own memory mapping
\begin{itemize}
 \item \texttt{sbrk()}
 \item \texttt{mmap()}
 \item \texttt{munmap(}
\end{itemize}

\end{frame}

 
\subsection{Weird stuff}
\begin{frame}{Ugly and evil syscalls}
A few nasty ones:
\begin{itemize}
 \item \texttt{ioctl()} \textbf{HERE THERE BE DRAGONS}
 \item \texttt{socket()}
 \item \texttt{send()}
 \item \texttt{recv()}
\end{itemize}

And approximately 300 more.
\end{frame}

\begin{frame}{``Non-syscall'' interfaces}
 Better API design than \texttt{ioctl} through a few different interfaces:
 \begin{itemize}
  \item \texttt{procfs}
  \item \texttt{sysfs}
 \end{itemize}
 
 \pause
 But these all eventually rely on \texttt{read()}/\texttt{write()}.
 
 \pause
 Even direct accesses to mapped IO space require a \texttt{mmap()} syscall to set up.
\end{frame}
 
\subsection{Security implications}
\begin{frame}{Kernel Attack Surface}
 Syscalls represent the kernel's attack surface from userspace (IO attack surface eg. ping-of-death is a different story)
 
 A naughty process can't exploit kernel bugs in syscalls it can't use.
\end{frame}

\begin{frame}{Userspace defense-in-depth}
 Restricting syscalls can also protect a process from itself.
 
 \pause
 Classic RCE:
 \begin{enumerate}
  \item Buffer overflow
  \item ROP or shellcode
  \item \texttt{execve(``/bin/sh'', null)}
  \item Profit!
 \end{enumerate}
 \pause
 
 Good luck with that if you can't use the
 \texttt{execve()} syscall!

\end{frame}

\begin{frame}{Sandboxing}
 \textbf{SECCOMP IS NOT A SANDBOX}
 \pause
 
 Intercepting or filtering syscalls can harden a sandbox:
 \pause
 \begin{itemize}
  \item Preventing opening new files
  \item Disallow execution of other executables
  \item Disallowing access to devices
  \item Aggressively whitelisting only \texttt{read()}s and \texttt{write()}s
  \item ...
 \end{itemize}

\end{frame}

\section{Seccomp}
\subsection{Seccomp overview}
\begin{frame}{Enter: seccomp}
 \texttt{seccomp} is the kernel feature that allows syscall filtering
 
 Matching rules and behavior (perform as normal, \texttt{EINVAL}, send a signal) set during the
 transition to seccomp mode - a one-way trip.
 
 Technically happens over a few different \texttt{prctl} syscalls to handle the processes
 privilege level and set up prerequisites to avoid breakout.
\end{frame}

\subsection{Modes of seccomp}
\begin{frame}{SECCOMP\_SET\_MODE\_STRICT}
Seccomp initially supported only ``strict'' mode.

Simple, well-defined behavior:
\begin{itemize}
 \item \texttt{read()}
 \item \texttt{write()}
 \item \texttt{\_exit()}
\end{itemize}
\pause

Nothing else allowed!
\pause
In addition to inflexibility, this totally breaks libc (hidden \texttt{sbrk()}s, \texttt{*at} syscalls, ...).
\end{frame}

\begin{frame}{SECCOMP\_SET\_MODE\_FILTER}
This is where all the cool stuff lives!


\begin{itemize}
 \item Flexibility with which calls to filter.
 \item Allow, crash, error, or signal
\end{itemize}
\pause
Signalling offers interesting options to emulate IO functionality for a sandbox\footcite{chromium}.
\end{frame}

\subsection{BPF Intro}
\begin{frame}{BPF intro}
 The BSD Packet Filter\footcite{bpfpaper} was designed for matching on binary data in network packets.
 \pause
 
 Classic BPF operates as a simple accumulator-based virtual machine
 \begin{itemize}
  \item Accumulator and index(x) registers
  \item Scratch memory
  \item Read-only data memory (the packet)
 \end{itemize}
\end{frame}

\begin{frame}
 Funny note: 
 
 SCO pointed to BPF as proof that Linux stole code from them...
 \pause despite SCO never owning BSD (and therefore the original BPF code) in the first place\footcite{scobullshit}.
\end{frame}

\begin{frame}[fragile]{BPF instruction encoding}
 Simple instruction encoding (two words per instruction):
 
 \begin{bytefield}[endianness=big]{32}
    \bitbox{16}{opcode:16} & \bitbox{8}{jt:8} & \bitbox{8}{jf:8} \\
    \bitbox{32}{k:32}
 \end{bytefield}
 
 Not all opcodes use jt, jf, and k
 
 jt and jf encode \textit{unsigned} jump distances for conditional jumps
 
 k encodes a constant
\end{frame}

\begin{frame}{Example opcodes}
 \begin{itemize}
  \item \texttt{ld[b|h]} (load from data to accumulator)
  \item \texttt{st} (store to scratch memory)
  \item \texttt{jne}, \texttt{jgt}, etc. (conditional jumps)
  \item \texttt{add}, \texttt{mul}, \texttt{or}, etc. (arithmetic and bitwise)
 \end{itemize}
\end{frame}

\begin{frame}{Address modes}
 Non-exhaustive list.
 \begin{itemize}
  \item Constant (encoded \texttt{k} in the instruction)
  \item \texttt{x}
  \item \texttt{[k]} or \texttt{[x+k]} (data)
 \end{itemize}
 \end{frame}
 
\begin{frame}{Example}
 \texttt{ldb [14]}\\
 \texttt{and \#0xf}\\
 \texttt{lsh \#2}\\
 \texttt{tax}\\
 \texttt{ldh [x+16]}\\
 \texttt{jeq \#80, 0, 1}\\
 \texttt{ret 1}\\
 \texttt{ret 0}\\
\end{frame}

\subsection{BPF for seccomp}
\begin{frame}{But how does it apply to seccomp!}
BPF programs are used to implement the logic for syscall filtering with seccomp.
\pause

The implementation is pretty basic:
\begin{itemize}
 \item ``Packet Data'' is a \texttt{struct seccomp\_data}
 \item BPF program return value determines behavior (allow/deny/EINVAL)
 \item Flexible logic:
 \begin{itemize}
  \item Masking operations to check flags in arguments
 \end{itemize}
  \item \textit{No} ability to deference pointers (avoid race conditions)
\end{itemize}

\pause
It is admittedly a bit overkill (eBPF is a story for another day).
\end{frame}

\begin{frame}[fragile]{\texttt{struct seccomp\_data}}
\begin{verbatim}
#include <seccomp.h>
struct seccomp_data {
    int     nr,                
    _u32    arch,           
    _u64    instruction_pointer,
    _u64    args[6]
};
\end{verbatim}

\end{frame}
\begin{frame}[fragile]{BPF Programs in code}
Classic BPF is easy enough to hand-write in C.

Macros and structs provided to make it easy to express a program:
 \begin{lstlisting}
  struct sock_filter filter[] = {
      BPF_STMT(BPF_LD + BPF_W + BPF_ABS, (offsetof(struct seccomp_data, arch))),
      BPF_JUMP(BPF_JMP + BPF_JEQ + BPF_K, FILTER_ARCH, 1, 0),
      BPF_STMT(BPF_RET + BPF_K, SECCOMP_RET_ERRNO | (EPERM & SECCOMP_RET_DATA)),
      BPF_STMT(BPF_LD + BPF_W + BPF_ABS, (offsetof(struct seccomp_data, nr))),
      BPF_JUMP(BPF_JMP + BPF_JEQ + BPF_K, __NR_read, 0, 1),
      BPF_STMT(BPF_RET + BPF_K, SECCOMP_RET_ALLOW),
      BPF_STMT(BPF_LD + BPF_W + BPF_ABS, (offsetof(struct seccomp_data, nr))),
      BPF_JUMP(BPF_JMP + BPF_JEQ + BPF_K, __NR_write, 0, 1),
      BPF_STMT(BPF_RET + BPF_K, SECCOMP_RET_ALLOW),
      BPF_STMT(BPF_LD + BPF_W + BPF_ABS, (offsetof(struct seccomp_data, nr))),
      BPF_JUMP(BPF_JMP + BPF_JEQ + BPF_K, __NR_writev, 0, 1),
      BPF_STMT(BPF_RET + BPF_K, SECCOMP_RET_ALLOW),
      BPF_STMT(BPF_LD + BPF_W + BPF_ABS, (offsetof(struct seccomp_data, nr))),
      BPF_JUMP(BPF_JMP + BPF_JEQ + BPF_K, __NR_open, 0, 1),
      BPF_STMT(BPF_RET + BPF_K, SECCOMP_RET_ALLOW),
      BPF_STMT(BPF_RET + BPF_K, SECCOMP_RET_ERRNO | (EPERM & SECCOMP_RET_DATA)),
  };
 \end{lstlisting}
\end{frame}

\subsection{Applications}
\begin{frame}{Applications}
 \begin{itemize}
  \item Chromium leverages seccomp for attack surface reduction
  \begin{itemize}
   \item also putting flash in time-out
  \end{itemize}
  \item Non-root sandboxing\footcite{mbox}
  \item Theoretical: mobile-style app sandboxing with native code
  \begin{itemize}
   \item Userspace platform API calls into ``trusted'' regions to do syscalls
   \item Sort of how VDSO and bits of NT already work
   \item ROP exists...
  \end{itemize}
  \item QEMU (attack surface reduction)
  \item Containers (magical christmas land)
 \end{itemize}
\end{frame}



\begin{frame}
 \printbibliography
\end{frame}



\end{document}
